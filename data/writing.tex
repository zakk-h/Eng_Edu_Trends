% Options for packages loaded elsewhere
\PassOptionsToPackage{unicode}{hyperref}
\PassOptionsToPackage{hyphens}{url}
\PassOptionsToPackage{dvipsnames,svgnames,x11names}{xcolor}
%
\documentclass[
  letterpaper,
  DIV=11,
  numbers=noendperiod]{scrartcl}

\usepackage{amsmath,amssymb}
\usepackage{iftex}
\ifPDFTeX
  \usepackage[T1]{fontenc}
  \usepackage[utf8]{inputenc}
  \usepackage{textcomp} % provide euro and other symbols
\else % if luatex or xetex
  \usepackage{unicode-math}
  \defaultfontfeatures{Scale=MatchLowercase}
  \defaultfontfeatures[\rmfamily]{Ligatures=TeX,Scale=1}
\fi
\usepackage{lmodern}
\ifPDFTeX\else  
    % xetex/luatex font selection
\fi
% Use upquote if available, for straight quotes in verbatim environments
\IfFileExists{upquote.sty}{\usepackage{upquote}}{}
\IfFileExists{microtype.sty}{% use microtype if available
  \usepackage[]{microtype}
  \UseMicrotypeSet[protrusion]{basicmath} % disable protrusion for tt fonts
}{}
\makeatletter
\@ifundefined{KOMAClassName}{% if non-KOMA class
  \IfFileExists{parskip.sty}{%
    \usepackage{parskip}
  }{% else
    \setlength{\parindent}{0pt}
    \setlength{\parskip}{6pt plus 2pt minus 1pt}}
}{% if KOMA class
  \KOMAoptions{parskip=half}}
\makeatother
\usepackage{xcolor}
\setlength{\emergencystretch}{3em} % prevent overfull lines
\setcounter{secnumdepth}{-\maxdimen} % remove section numbering
% Make \paragraph and \subparagraph free-standing
\ifx\paragraph\undefined\else
  \let\oldparagraph\paragraph
  \renewcommand{\paragraph}[1]{\oldparagraph{#1}\mbox{}}
\fi
\ifx\subparagraph\undefined\else
  \let\oldsubparagraph\subparagraph
  \renewcommand{\subparagraph}[1]{\oldsubparagraph{#1}\mbox{}}
\fi


\providecommand{\tightlist}{%
  \setlength{\itemsep}{0pt}\setlength{\parskip}{0pt}}\usepackage{longtable,booktabs,array}
\usepackage{calc} % for calculating minipage widths
% Correct order of tables after \paragraph or \subparagraph
\usepackage{etoolbox}
\makeatletter
\patchcmd\longtable{\par}{\if@noskipsec\mbox{}\fi\par}{}{}
\makeatother
% Allow footnotes in longtable head/foot
\IfFileExists{footnotehyper.sty}{\usepackage{footnotehyper}}{\usepackage{footnote}}
\makesavenoteenv{longtable}
\usepackage{graphicx}
\makeatletter
\def\maxwidth{\ifdim\Gin@nat@width>\linewidth\linewidth\else\Gin@nat@width\fi}
\def\maxheight{\ifdim\Gin@nat@height>\textheight\textheight\else\Gin@nat@height\fi}
\makeatother
% Scale images if necessary, so that they will not overflow the page
% margins by default, and it is still possible to overwrite the defaults
% using explicit options in \includegraphics[width, height, ...]{}
\setkeys{Gin}{width=\maxwidth,height=\maxheight,keepaspectratio}
% Set default figure placement to htbp
\makeatletter
\def\fps@figure{htbp}
\makeatother

\KOMAoption{captions}{tableheading}
\makeatletter
\makeatother
\makeatletter
\makeatother
\makeatletter
\@ifpackageloaded{caption}{}{\usepackage{caption}}
\AtBeginDocument{%
\ifdefined\contentsname
  \renewcommand*\contentsname{Table of contents}
\else
  \newcommand\contentsname{Table of contents}
\fi
\ifdefined\listfigurename
  \renewcommand*\listfigurename{List of Figures}
\else
  \newcommand\listfigurename{List of Figures}
\fi
\ifdefined\listtablename
  \renewcommand*\listtablename{List of Tables}
\else
  \newcommand\listtablename{List of Tables}
\fi
\ifdefined\figurename
  \renewcommand*\figurename{Figure}
\else
  \newcommand\figurename{Figure}
\fi
\ifdefined\tablename
  \renewcommand*\tablename{Table}
\else
  \newcommand\tablename{Table}
\fi
}
\@ifpackageloaded{float}{}{\usepackage{float}}
\floatstyle{ruled}
\@ifundefined{c@chapter}{\newfloat{codelisting}{h}{lop}}{\newfloat{codelisting}{h}{lop}[chapter]}
\floatname{codelisting}{Listing}
\newcommand*\listoflistings{\listof{codelisting}{List of Listings}}
\makeatother
\makeatletter
\@ifpackageloaded{caption}{}{\usepackage{caption}}
\@ifpackageloaded{subcaption}{}{\usepackage{subcaption}}
\makeatother
\makeatletter
\@ifpackageloaded{tcolorbox}{}{\usepackage[skins,breakable]{tcolorbox}}
\makeatother
\makeatletter
\@ifundefined{shadecolor}{\definecolor{shadecolor}{rgb}{.97, .97, .97}}
\makeatother
\makeatletter
\makeatother
\makeatletter
\makeatother
\ifLuaTeX
  \usepackage{selnolig}  % disable illegal ligatures
\fi
\IfFileExists{bookmark.sty}{\usepackage{bookmark}}{\usepackage{hyperref}}
\IfFileExists{xurl.sty}{\usepackage{xurl}}{} % add URL line breaks if available
\urlstyle{same} % disable monospaced font for URLs
\hypersetup{
  pdftitle={STA 210 - Final Project},
  pdfauthor={Zakk Heile \& Julia Healey-Parera},
  colorlinks=true,
  linkcolor={blue},
  filecolor={Maroon},
  citecolor={Blue},
  urlcolor={Blue},
  pdfcreator={LaTeX via pandoc}}

\title{STA 210 - Final Project}
\author{Zakk Heile \& Julia Healey-Parera}
\date{}

\begin{document}
\maketitle
\ifdefined\Shaded\renewenvironment{Shaded}{\begin{tcolorbox}[interior hidden, borderline west={3pt}{0pt}{shadecolor}, frame hidden, breakable, enhanced, sharp corners, boxrule=0pt]}{\end{tcolorbox}}\fi

\textbf{Introduction}

\textbf{Project motivation + research question}

It is no surprise that early childhood education affects eventual
employment. Current research supports the theory that the quality of
education received during primary and secondary school has a strong
positive relationship with wages earned later in life (Lee \& Lee,
2024). Operating under the assumption that this research is accurate, we
can thus assume that proper prediction of educational attainment in
primary and secondary schooling would allow for estimation of earnings
later in life. Thus, this study builds upon prior research and attempts
to create a model to predict educational attainment given indicators
grouped by regional location (i.e.~town).~

\textbf{Dataset explanation}

In order to achieve our goal of creating a model for predicting
educational attainment, we used a dataset sourced from TidyTuesday and
The UK Office for National Statistics. The UK Office for National
Statistics is the recognized statistical institution of the nation that
carries out the census for England and Wales in addition to the
collection of a multitiude of other data made publicly available. The
selected dataset, titled ``Educational attainment of young people in
English towns'' details the educational score of each town in the UK
using attainment scores from the 2012 key stage 4 cohort of that town.
Key stage 4, which is the American equivalent of freshman and sophomore
year, are the two years in which students (typically aged 14-16) study
for and take General Certificate of Secondary Education (GCSE) exams.
Thus, our outcome variable is a level of educational attainment at a
time of standard evaluation for students in the UK with an equal
metric.~

\textbf{Relevant variables}

The variables included in our final model were tested for statistical
significance as related to the outcome variable of educational
attainment. This is outlined in our methodology section. The five final
relevant variables are as follows:~

\begin{itemize}
\item
  population\_2011 -~
\item
  Highest\_level\_qualification\_achieved\_b\_age\_22\_average\_score -
\item
  level\_3\_at\_age\_18 -~
\item
  activity\_at\_age\_19\_employment\_with\_earnings\_above\_10\_000 -
\item
  key\_stage\_4\_attainment\_school\_year\_2012\_to\_2013 -
\end{itemize}

Lee, Hanol \& Lee, Jong-Wha. (2024). Educational quality and disparities
in income and growth across countries. Journal of Economic Growth. 1-29.
10.1007/s10887-023-09239-3.~

In our dataset, education score ranges from -10.03 to 11.87, population
ranged from 5003 to 1085810, highest qualification ranges from 2.57 to
5.14, percent obtained level 3 ranged from 16.54 to 85.71, percent
earning \textgreater10,000 pounds at 19 ranged from 7.06 to 48.98, and
percent of students that achieved 5 or more C or higher certificates
ranged from 33.33 to 92.86.

Complete case analysis -- get rid of all observations with NAs for one
or more variables that we are using in our model.~

Link to dataset:
\url{https://github.com/rfordatascience/tidytuesday/blob/master/data/2024/2024-01-23/readme.md}

\url{https://github.com/zakk-h/STA210Final}

Predict key\_stage\_4\_attainment\_school\_year\_2012\_to\_2013 or
highest\_level\_qualification\_achieved\_by\_age\_22\_level\_6\_or\_above
(proportion) - need Beta regression or transform response into log odds
and transform back after

Predict activity\_at\_age\_19\_full\_time\_higher\_education - logistic

Predict logistic -
activity\_at\_age\_19\_employment\_with\_earnings\_above\_10\_000

Population\_2011 vs.~size\_flag~

Income\_flag vs.~job\_density\_flag

rgn11nm vs coastal

Highest\_level\_qualification\_achieved\_b\_age\_22\_average\_score

\begin{longtable}[]{@{}l@{}}
\toprule\noalign{}
\endhead
\bottomrule\noalign{}
\endlastfoot
 \\
\end{longtable}

Explain the reasoning for the type of model you're fitting, predictor
variables considered for the model including any interactions

Methodology/Final Model Results

Our linear model's R-squared metric (both multiple and adjusted) is
above 96\%. Over 96\% of the variation in education scores are explained
by our model.~

The residual standard error, a measure of the standard deviation of the
residuals from the predicted values, is 0.7.

The F-statistic has a p-value of 2.2*10\^{}-16, which is highly
statistically significant. We can reject the null hypothesis of all
coefficients being 0 with extremely high confidence.

The residual plot shows great symmetry across the residual = 0 line,
indicating that the linearity assumption in the parameters is indeed
met. Visually, homoscedasticity looks reasonable, but when fitted values
are 0, there is more variance. Code was used to find 11 quantiles, split
the observations into deciles based on those quantiles, and then
calculate the variance of the residuals for each decile. From there, we
quantitatively assessed homoscedasticity, and each decile's variance was
reasonably close to all others, with the exception being where the
fitted values are 0. The Breusch-Pagan test finds statistically
significant evidence for heteroscedasticity. However, we moved forward
with this model because the heteroscedasticity does not bias the
estimates for the coefficients. We also explored transformations in the
predictors but did not find them worth pursuing.~

The quantile-quantile plot shows that the residuals roughly follow a
normal distribution, with the left tail deviating slightly and seeming
heavier, but when combined with the histogram, appear sufficiently
normal for our purposes.

Independence of residuals: The observations are unique per region,
meaning no regions are double counted and each is disjoint, so the
observations are fully distinct from one another. Additionally, the data
is not collected over time, which could be a problem if the observations
overlapped. Thus, the errors are independent, which implies the
predicted values are independent.~

We predicted an aggregate education score that ranged between -10.028
and 11.872 in our dataset. Our predictors were all untransformed:
population of a town/city + average of the highest level of
qualification at age 22, the proportion of employed 19-year-olds in that
town/city earning more than 10,000 pounds, the proportion of students in
that town/city earning more than 5 certificates with grades no less than
C. All of our coefficients are statistically significant at the 0.05
level. All predictors except for population increased education scores
when each one was increased on its own, holding the other predictors
constant. The population slope is negative, though it is very small, is
for a one-person increase in population. The slope is much more
meaningful when you consider a larger increase like one thousand
people.~~~

\textbf{Discussion}

\textbf{VARIABLE DISCUSSION JULIA}

\begin{itemize}
\item
  Summary -- what is the impact of each coefficient upon educational
  attainment and why?~

  \begin{itemize}
  \item
    Highest\_level\_qualification\_achieved\_b\_age\_22\_average\_score

    \begin{itemize}
    \tightlist
    \item
      Higher later educational attainment -\textgreater{} more likely~
    \end{itemize}
  \item
    Activity\_at\_age\_19\_employment\_with\_earnings\_above\_10\_000

    \begin{itemize}
    \item
      Lower levels of educational attainment in urban areas / more
      poverty in urban areas -- higher levels of educational attainment
      in rural areas / less poverty in rural areas / by association,
      higher levels of education in areas of less poverty

      \begin{itemize}
      \tightlist
      \item
        Barriers to getting a job bc of other factors?
      \end{itemize}
    \item
      Internships pay? More qualified o
    \item
      Discussed this conflict with students from the UK\ldots{} found
      evidence to be in contradiction with observed reality -- explain
      this?~
    \item
      Rate for apprentices -- 2.6 pounds an hour in 2011

      \begin{itemize}
      \item
        Rate for 18-20 year olds -- 4.98 pounds an hour~
      \item
        Earnings above 10,000 -\textgreater{} working job full time that
        is NOT an apprenticeship (not vocational -\textgreater{} working
        other job in preparation for university or later career)~

        \begin{itemize}
        \tightlist
        \item
          More likely to have higher educational attainment if working
          job in prep for uni
        \end{itemize}
      \end{itemize}
    \end{itemize}
  \item
    Level\_3\_at\_age\_18

    \begin{itemize}
    \tightlist
    \item
      The percent that obtained level 3 qualifications (equivalent to
      the first year of a US college degree)
    \end{itemize}
  \item
    Key\_stage\_4\_attainment\_school\_year\_2012\_to\_2013

    \begin{itemize}
    \tightlist
    \item
      The percent of students that passed (C-) at least 5 of their
      subject exams (baseline for less selective colleges).~
    \end{itemize}
  \item
    Population goes into next point
  \end{itemize}
\item
  Why is population coefficient negative?~

  \begin{itemize}
  \item
    Unique to UK\ldots{} why?~

    \begin{itemize}
    \item
      Reference other article dataset was used in~
    \item
      Educational attainment higher in urban areas/larger towns in USA~
    \item
      Conclusions and model unique to UK -\textgreater{} flow into
      limitations of model
    \end{itemize}
  \end{itemize}
\end{itemize}

\textbf{OTHER LIMITATIONS + FUTURE RESEARCH ZAKK}

~Our data potentially capturs individuals' locations at the time of
surveying, which could assign them to towns different from where they
received their earlier education, notably year 13 and earlier, which we
are interested in. Two potential causes of this would be relocating for
work or college. Whether or not this is a limiting factor depends on how
exactly the data was collected.

Additionally, the economic variable we used in the model, the percentage
of individuals earning above £10,000 at age 19, is grounded in the
context of 2012's economy. If the same variable was used today, given
inflationary trends, it would change the meaning of the variable because
apprenticeships could now be above the cutoff, when as of the 2012 data,
only those that have high-paying jobs at age 19 would be included.
High-paying jobs we argue are associated with a higher level of
education, and our data supports this.~

\begin{verbatim}
Future research could investigate how individuals perform in college (if they go) to see how well their earlier studies prepared them for it. Additional variables that, if there were in the dataset, would aid in predicting education score are: average income in area, student:teacher ratio, school funding per pupil, and poverty percentage. We are fascinated by the reason that an increase in population is associated with a decrease in education scores, which is in contrary to our intuition for the United States education system, and believe it is worth additional exploration. Future research could deploy a multinomial model to predict regions or types of towns/cities based on education statistics as well, which would target a different question than the one we are addressing, but an interesting one nonetheless. 
\end{verbatim}



\end{document}
